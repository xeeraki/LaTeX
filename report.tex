%++++++++++++++++++++++++++++++++++++++++
% Don't modify this section unless you know what you're doing!
\documentclass[letterpaper,12pt]{article}
\usepackage{tabularx} % extra features for tabular environment
\usepackage{amsmath}  % improve math presentation
\usepackage{graphicx} % takes care of graphic including machinery
\usepackage[margin=1in,letterpaper]{geometry} % decreases margins
\usepackage{cite} % takes care of citations
\usepackage[final]{hyperref} % adds hyper links inside the generated pdf file
\usepackage{caption}
\usepackage{tocloft}
\usepackage{}

\hypersetup{
	colorlinks=true,       % false: boxed links; true: colored links
	linkcolor=blue,        % color of internal links
	citecolor=blue,        % color of links to bibliography
	filecolor=magenta,     % color of file links
	urlcolor=blue         
}
%++++++++++++++++++++++++++++++++++++++++


\begin{document}
\title{Measurements of IV and CV characteristic of Si and SiC pn-diodes\\
IH1611 lab report KTH}
\author{Adam Shafai\\ \href{mailto:hshafai@kth.se}{hshafai@kth.se}}



\date{\today}
\clearpage
\maketitle
\thispagestyle{empty}

\newpage
\tableofcontents
\newpage
\begin{abstract}
In this report the current-voltage IV and capacitance-voltage CV characteristic of pn-junction of silicon Si and silicon carbide SiC are investigated experimentally. The experiment is performed at room temperature and $50\,^{\circ}\mathrm{C}$ with and without light. The objective of this experiment is to study the effect of ambient conditions such as light and temperature on the diodes characteristic and to extract the physical parameter from the sample such as doping level and model parameters such as revers saturation current $I_0$,ideality factor $n$,parasitic resistance $r_s$ and voltage over pn-junction $V_d$ . A probe station together with IV-analyzer and CV-meter is used for the experiment and the measured data is fitted to extract the model parameter. The study showed that light and temperature has significant effect on the diodes IV characteristic of Si. Light had almost negligible effect on SiC:s IV-characteristic but temperature had some small effect on SiC IV characteristic. The study showed that light and temperature had much smaller effect on SiC compare to the Si ,this is because of that SiC has larger energy band gap compared to Si.The ambient light has no effect on CV characteristic of both Si and SiC but temperature had some effect on CV of both Si and SiC. The extracted model parameter showed to be $ n = 2.80$, $r_s = 9.8\Omega$, $I_0 = 3.7*10^{-5} mA$ and $V_d = 1.07V$.
The doping concentration showed to be $N_d = 5.5*10^{16} cm^{-3}$
\end{abstract} 


\section{Theory}
A pn-junction is formed when P-type semiconductor material is joined together with N-type semiconductor material~\cite{hu}. when the N-type and P-type material are joined the free electron will  diffuse from the N-region to the P-region and leave positive charged donor ion $N_d$ behind also the hole from P-region will diffuse across the junction toward N-region and leave negative acceptor ion $N_a$ behind~\cite{elec}.This process will continue until equilibrium state is reached and this produce a potential barrier between P-region and N-region.This process form a new region between N-region and
P-region which calls depletion region as showed in figure 1 and it has a built in potential which showed in figure 2 ~\cite{elec}.

\begin{minipage}{\linewidth}% to keep image and caption on one page
\makebox[\linewidth]{%        to center the image
  \includegraphics[keepaspectratio=true,scale=0.3]{depletion.png}}
\captionof{figure}{pn-junction with it's three region shown ~\cite{hu}}\label{visina8}%      only if needed  
\end{minipage}

\begin{minipage}{\linewidth}% to keep image and caption on one page
\makebox[\linewidth]{%        to center the image
  \includegraphics[keepaspectratio=true,scale=0.3]{builtinpo.png}}
\captionof{figure}{Energy band diagram of pn-junction and built in potential at zero bias ~\cite{hu}}\label{visina8}%      only if needed  
\end{minipage}\bigskip

When a positive voltage applied to the N region of a pn-junction relative to the P region than it's revers biased and this will also increase the potential height from $\phi_{bi}$ to $\phi_{bi} + V$. The depletion region together with N and P region act as parallel plate capacitor which it's capacitance is given by the equation 1~\cite{hu}.
\begin{equation}
    C_{dep} = A\frac{\epsilon_s}{W_{dep}}
\end{equation}
$C_{dep}$ is the depletion layer capacitance, A is the area of pn-junction $\epsilon_s$ is the dielectric constant, $W_{dep}$ is the depletion layer width.The
depletion layer capacitance-voltage CV-curve is of great importance and can be used to find the doping concentration and built in potential.There is a linear relationship between $\frac{1}{C^2}$ and revers bias voltage $V_r$  from the slope of the linear graph one can determine the doping concentration and from the interception of the straight line with the $\frac{1}{C^2}$ axis one can determine the built in potential~\cite{hu}.
\bigskip

The pn-diodes IV characteristic is of great importance in order to study the behavior of the pn-diode.The diode IV model is given by the following equation.

\begin{equation}
    I_D = I_0\bigg(exp\Big(\frac{q(V-I_DR_s)}{nkT}\Big)\bigg)
\end{equation}
$I_0$ is the revers saturation current , $V$ is the voltage over pn-junction and $n$ is ideaality factor~\cite{lab}. With increasing forward bias there will be a potential drop and this potential drop can be modeled by a resistance placed in series with the diode~\cite{lab}.The extraction of the model parameter such as $R_s$ ,$I_0$ and $n$ can be done by linear fitting showed in section analysis.



\section{Procedures}
This experiment were done in a group of four student under guidance of laboratory assistant at KTH, Department of Electronics in Kista. A probe station equip with microscopy camera connected to IV-analyzer and CV-meter is used to perform the experiment. First the silicon Si wafer was carefully placed at the probe stations wafer-holder than with the help of the camera adjusted the wafer,to make sure that it's fully connected.
The first and second measurements were IV-characteristic of Si over $-1.5-2 V$ voltage range at room temperature with and without light.The third and forth measurements were CV-sweep of Si with the same temperature with and without light.The same measurement were continued for Si but this time raising the temperature to  $50\,^{\circ}\mathrm{C}$. For the silicon carbide SiC the measurements were done similar to the measurements for Si but with different voltage range $(-2-8 V)$.The measured data were saved as Excel spreadsheet in the IV-analyzer which later sent by teacher to each participant student for analyze. MATLAB is used to analyze and plot the collected data and the result and plots are present in this report.
\section{Result}
\subsection{IV-Measurements}
The measurements of IV characteristic of Si and SiC wafer at different temperatures with and without light.

\noindent%
 \begin{minipage}[b]{0.48\textwidth}
    \includegraphics[width=\textwidth]{siivcorp.pdf}
    \captionof{figure}{IV characteristic of Si at RT\newline and $50\,^{\circ}\mathrm{C}$ with and without light.}\label{visina8}%      only if needed  
    \label{fig:1}
\end{minipage}
   %
\begin{minipage}[b]{0.48\textwidth}
    \includegraphics[width=\textwidth]{sicivcorp.pdf}
     \captionof{figure}{IV characteristic of SiC at RT and $50\,^{\circ}\mathrm{C}$ with and without light.}\label{visina8}%      only if needed  
    \label{fig:2}
\end{minipage}

\noindent%
\begin{minipage}[b]{0.48\textwidth}
    \includegraphics[width=\textwidth]{sicivzoomcorp.pdf}
   \captionof{figure}{Enlarged version of figure 4.}\label{visina8}%      only if needed  
    \label{fig:2}
\end{minipage}



\subsection{CV-Measurements}
The measurements of CV characteristic of Si and SiC wafer at different temperatures with and without light.

\noindent%
  \begin{minipage}[b]{0.5\textwidth}
    \includegraphics[width=\textwidth]{cvsicorp.pdf}
   \captionof{figure}{CV characteristic of Si at RT\newline and $50\,^{\circ}\mathrm{C}$ with and without light.}\label{visina8}%      only if needed  
    \label{fig:3}
  \end{minipage}
  %
  \begin{minipage}[b]{0.5\textwidth}
    \includegraphics[width=\textwidth]{siccvcorp.pdf}
    \captionof{figure}{CV characteristic of SiC at RT\newline and $50\,^{\circ}\mathrm{C}$ with and without light.}\label{visina8}%      only if needed  
    \label{fig:4}
 \end{minipage}
 
 \noindent%
  \begin{minipage}[b]{0.5\textwidth}
    \includegraphics[width=\textwidth]{cvsizoomcorp.pdf}
   \captionof{figure}{Enlarged version of figure 6}\label{visina8}%      only if needed  
    \label{fig:3}
  \end{minipage}
  %
  \begin{minipage}[b]{0.5\textwidth}
    \includegraphics[width=\textwidth]{siccvzoomcorp.pdf}
    \captionof{figure}{Enlarged version of figure 7}\label{visina8}%      only if needed  
    \label{fig:4}
 \end{minipage}
 
 
\section{Analysis}
\subsection{Fitting model parameters}
The data points in table~\ref{tbl:bins} were collected from the figure 10 these value are used to extract the model parameters $I_0, r_s $ and $n$. 
\bigskip

\noindent%
\begin{minipage}{\linewidth}% to keep image and caption on one page
\makebox[\linewidth]{%        to center the image
  \includegraphics[keepaspectratio=true,scale=0.7]{ivfitcorp.pdf}}
\captionof{figure}{IV of Si for small voltage range to collect data points for Linearazation}\label{visina8}%      only if needed  
\end{minipage}

\begin{center}
\captionof{table}{Linearization of Current-Voltage data }
\label{tbl:bins} % spaces are big no-no withing labels
\begin{tabular}{|ccc|} 
\hline
\multicolumn{1}{|c}{$I_d$ (mA)} & \multicolumn{1}{|c|}{$x = V_d$ (V)}& \multicolumn{1}{c|}{$y = ln(I_d)$ (mA)}  \\
\hline
0.14 &   0.62  & -1.96\\
0.26 &   0.65 & -1.35\\
0.45 &   0.68 & -0.80\\
0.95 &   0.72 & -0.05\\
\hline
\end{tabular}
\end{center}
For small current $\leq1mA$ the parasitic resistance can be neglected and the diode equation can be linearized as following~\cite{sup}.


\begin{equation}
\ln(I_d) = ln(I_0) + \frac{1}{nV_t}V_d 
\end{equation}

\begin{equation}
    y = a_0 +a_1x
\end{equation}


\begin{equation}
\begin{aligned}
na_0 + (\sum\limits_{i=1}^n x_i)a_1 = \sum\limits_{i=1}^n y_i & \\
(\sum\limits_{i=1}^n x_i)a_0 + (\sum\limits_{i=1}^n x_i^2)a_1 = \sum\limits_{i=1}^n y_ix_i
\end{aligned}
\end{equation}
From the the table~\ref{tbl:bins} we see that there are four data points are 
collected from the Figure 10 and this gives $n = 4$. Calculating the data points from the table~\ref{tbl:bins} and plugging the value to the system of equation (5) gives the following linear system of equation.
\begin{equation}
\begin{aligned}
    4a_0 + 2.67a_1 = -4.16 &  \\
    2.67a_0 + 1.79a_1 = -2.67 & 
\end{aligned}
\end{equation}
Solving the system of equation (6) obtained that the coefficient $ a_0 = -10.20$ and $a_1 = 13.74$
and the revers saturation current become
$I_0 = exp(a_0) = 3.7*10^{-5}  m A $  
and the ideality factor 
$n = \frac{1}{a_1V_t} = 2.80 $ 
\newline 
If the parasitic  resistance $r_s$ was zero, voltage $V_d$ at
highest $I_d$ = 95mA would be as 
\begin{equation}
    V_d = 2.80*0.026ln\Big(\frac{95}{3.7*10^{-5}}\Big) = 1.07 V
\end{equation}
 The measured voltage was 2.0 V and the difference between measured voltage and $V_d$ at highest $I_d$ is   
$V = 2.0-1.07 = 0.93V$ and this lead to that the resistance 
$r_s = 9.8\Omega$


\subsection{Substrate doping}
There is a linear relationship between $\frac{1}{C^2}$ and the breakdown voltage $V_r$~\cite{hu}.With the help of the slope of the figure:11 $N_d$ can be determined by the equation (8).
The diodes area was given as $A = 0.255*10^{-3} cm^{-3}$

\begin{equation}
    N_d = \frac{2}{slope*q\epsilon_sA^2}
\end{equation}

The $slope$ was determined from the figure:11 by choosing four arbitrary points and plugged into the equation (8) which gives the doping concentration 
$N_d = 5.5*10^{16} cm^{-3}$


\begin{minipage}{\linewidth}% to keep image and caption on one page
\makebox[\linewidth]{%        to center the image
  \includegraphics[keepaspectratio=true,scale=0.7]{cvsislopecorp.pdf}}
\captionof{figure}{CV characteristic of Si at RT with light}\label{visina8}%      only if needed  
\end{minipage}
\subsection{Influence of ambient conditions}
From the figure 3 we see that light has significant effect on the IV characteristic of Si. With light rise the current curve significant and light also make that breakdown happen in higher voltage. Changing the temperature from room temperature to $50\,^{\circ}\mathrm{C}$ shift the IV curve to lower voltage as we see in figure 3 the red curve shift to the left compare to the green curve.
From the figure 4 and 5 IV characteristic of SiC we see that light has small effect on SiC but it's almost insignificant compare to the Si in other hand the temperature has some effect which we see the IV curve shift left to the lower voltage.

The CV-plots in figure 8 and 9 which are enlargement portion of figure 6 and 7 shows that light has no influence at all on the capacitance of both Si and SiC.

Rising the temperature to $50\,^{\circ}\mathrm{C}$ increase the capacitance of Si and SiC as showed in figure 6 and 7.


\section{Conclusions}
The study showed that light and temperature has significant influence on Si which can be explained that the visible light spectrum is between 0.5 and 0.7 $\mu$m and the Si have a band gap corresponding $hv$ of infrared light that's because it absorb visible light strongly~\cite{hu}. The study showed that with higher temperature shifts the IV curve to the left which make the diode to operate at lower voltage. SiC showed to be more stable compare to the Si, both the light and temperature has little influence on SiC, this can be explained because of that SiC has larger(3.2 4H SiC) energy band gap compared to the Si(1.12)~\cite{sic}.The study showed that light has no effect on the the CV-characteristic of Si and SiC but temperature has some effect on both Si and SiC:s CV-characteristic.

%++++++++++++++++++++++++++++++++++++++++
% References section will be created automatically 
% with inclusion of "thebibliography" environment
% as it shown below. See text starting with line
% \begin{thebibliography}{99}
% Note: with this approach it is YOUR responsibility to put them in order
% of appearance.
\begin{thebibliography}{99}

\bibitem{hu}
Hu, C. (2010). Modern semiconductor devices for integrated circuits. Upper Saddle River, N.J.: Prentice Hall.

\bibitem{elec}
\url{https://www.electronics-tutorials.ws/diode/diode_2.html}

\bibitem{lab}
IH1611 Laboratory exercise paper KTH .
\bibitem{sup}
IH1611 Supplementary LAB how to extract $R_s,I_0,n$ KTH.
\bibitem{sic}Zetterling, C. (2002). Process technology for silicon carbide devices.
\end{thebibliography}


\end{document}
